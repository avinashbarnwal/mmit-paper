\documentclass{article}

\usepackage{natbib}
\usepackage[cm]{fullpage}
\usepackage{tikz}
\usepackage{algorithm,algorithmic}
\usepackage{amssymb,amsmath,amsthm}
\newtheorem{theorem}{Theorem}
\newtheorem{lemma}{Lemma}
\newtheorem{definition}{Definition}
\newtheorem{example}{Example}
\DeclareMathOperator*{\argmin}{arg\,min}
\DeclareMathOperator*{\argmax}{arg\,max}
\DeclareMathOperator*{\minimize}{minimize}

\begin{document}

\section{Introduction}
Interval regression is a class of machine learning models for which
the predicted value is a real number, but each output in the training
data is an interval $(\underline y_i, \overline y_i)$.  In the
terminology of the survival analysis literature, outputs may be either
left-censored ($-\infty=\underline y_i<\overline y_i<\infty$),
right-censored ($-\infty<\underline y_i<\overline y_i=\infty$), or
interval-censored ($-\infty<\underline y_i<\overline y_i<\infty$).
There are surprisingly few existing models that are designed to learn
from these outputs, and most are linear models. In this paper we
explore a nonlinear decision tree model for interval regression.


Interval regression models are useful in several
kinds of real data sets. One example is in data sequences measured
over space or time, with labels for presence or absence of
changepoints in particular data subsets \citep{HOCKING-penalties}. In
this context, learning a penalty function for predicting the number of
changepoints is an interval regression problem where each output is an
interval of penalty values with minimal incorrect labels.

\subsection{Contributions and organization}

Our first contribution is Section~\ref{sec:model}, in which we propose
a new nonlinear decision tree model for the interval regression
problem. We propose to choose splits in the tree model using a
margin-based hinge loss, which yields sequence of $n-1$ related convex
optimization problems for each feature. Our second contribution is
Section~\ref{sec:algorithm}, in which we propose a dynamic programming
algorithm that computes the optimal solution to all $n-1$ problems in
$O(n\log n)$ time. In Section~\ref{sec:results} we show that our
algorithm achieves state-of-the-art prediction accuracy in several
real and simulated data sets. In Section~\ref{sec:discussion} we
discuss the significance of our contributions, and in
Section~\ref{sec:conclusions} we propose possible future research
directions.

\section{Related work}

A related problem is linear models for censored outputs, which has
been extensively studied in the survival data analysis literature
\citep{Wei1992}, under the name ``accelerated failure time models.''
A distribution such as the logistic is assumed for the output value,
and the cumulative distribution function replaces the density in the
maximum likelihood inference problem \citep{log-logistic}. For
unregularized models with few features, second-order optimization
methods are typically used. 

L1-regularized accelerated failure time models have recently been
proposed in order to perform automatic feature selection in high
dimensional data \citep{cai-regularized-aft,
  huang-regularized-aft}. First-order optimization methods are
typically used. 

Nonlinear survival models are a relatively new research area,
including models such as Support Vector Machines \citep{kssvm} and
random forests \citep{survival-ensembles}. However, most models are
limited to the case of right-censored and un-censored outputs. In
contrast, in the interval regression setting that we consider, there
are no un-censored outputs. The only existing nonlinear model for this
setting is the recently proposed transformation forest of
\citet{transformation-forests}.

\section{Problem}
Let $y_1,\dots,y_n\in\mathbb R$ and $s_1,\dots,s_n\in\{-1,1\}$ be the
limits and their signs ($s_i=-1$ for a lower limit and $s_i=1$ for an
upper limit). Let $\epsilon\in\mathbb R$ be a margin parameter, and
let $\phi(x)=\ell[(x)_+]$ be a hinge loss, where $(x)_+$ is the
positive part function. We will explore two hinge loss functions: the
linear hinge loss $\ell(x)=x$, and the squared hinge loss
$\ell(x)=x^2$.

To choose the best split at any given step in the tree model fitting
algorithm, we need to find the predicted values below
$\mu_1\in\mathbb R$ and above $\mu_2\in\mathbb R$ a threshold
$t\in\{1,\dots,n-1\}$ (assume the outputs $y_1,\dots,y_n$ are sorted
with respect to an input variable $x_1<\cdots x_n$).  The total hinge
loss up to data point $t$ is
\begin{equation}
  \label{eq:underline_C}
  \underline C_t(\mu) = \sum_{i=1}^t \phi[s_i(\mu-y_i)+\epsilon],
\end{equation}
and the total hinge loss after data point $t$ is
\begin{equation}
  \label{eq:overline_C}
  \overline C_t(\mu) = \sum_{i=t+1}^n \phi[s_i(\mu-y_i)+\epsilon].
\end{equation}
Note that $\underline C_t$ and $\overline C_t$ are convex, because
each is defined as the sum of convex functions. But there are $n-1$
possible splits, so the search for the best split is a discrete
optimization problem:
\begin{equation}
  \label{eq:min_hinge_loss}
  \min_{
    t\in\{1,\dots,n-1\},
    \mu_1\in\mathbb R,
    \mu_2\in\mathbb R
  }
  \underline C_t(\mu_1) + \overline C_t(\mu_2) = 
  \min_{
    t\in\{1,\dots,n-1\}
  }
  \underline{ C}^*_t + \overline C^*_t.
\end{equation}
In the next section we propose a dynamic programming algorithm for
computing the vector of optimal cost
$\underline{ C}^*_1,\dots,\underline{ C}^*_n$ values.

\section{Algorithm}

The optimal cost function $\underline C_t(\mu)$ is piecewise
quadratic, and can be recursively computed using a dynamic programming
algorithm. First we define how we will store these functions, and then
we provide the dynamic programming update rules.

\subsection{Definition and storage of piecewise quadratic functions}
We will write and store the cost in terms of quadratic
functions:
\begin{equation}
  \label{eq:F}
  \mathcal F = \{
  f:\mathbb R \rightarrow \mathbb R\mid
  f(\mu) = a\mu^2 + b\mu + c 
  \text{ for some }a,b,c\in\mathbb R
  \}
\end{equation}
Note that $\ell\in\mathcal F$ but $\phi\not\in\mathcal F$, so
$\underline C_t\not\in \mathcal F$. In order to describe the piecewise
function $\underline C_t$, we first define the space of all
possible breakpoints in such functions:
\begin{equation}
  \label{eq:B^t}
  \mathcal B^t = \{
  (b,f)\in\mathbb R^t\times \mathcal F^t \mid
  b_1 < \cdots < b_t
  \}.
\end{equation}

\begin{figure*}
  \hskip -0.3cm
  \input{figure-algorithm-steps}
  \vskip -0.8cm
  \caption{First two steps of the dynamic programming algorithm for
    the data $y_1=4, s_1=1, y_2=1, s_2=-1$ and margin $\epsilon=1$
    using the linear hinge loss $\ell(x)=x$. \textbf{Left:} the
    algorithm begins by creating a first breakpoint at
    $b_{1,1}=y_1-\epsilon=3$, with corresponding function
    $f_{1,1}(\mu)=\mu-3$. The algorithm also stores the cost function
    $m_1(\mu)=\mu-3$ on the interval to the left of the pointer
    $j_1=2$. Note that the cost $c_{1,1}$ before the first breakpoint
    is not yet stored by the algorithm. \textbf{Middle:} the
    optimization step is to move the pointer to the minimum
    ($J_1=j_1-1$), and update the cost function,
    $M_1(\mu)=m_1(\mu) -f_{1,1}(\mu)$.  \textbf{Right:} the algorithm
    adds the second breakpoint at $b_{2,1}=y_2+\epsilon=2$ with
    $f_{2,1}(\mu)=\mu-2$. The cost at the pointer is not affected by
    the new data point, so the pointer does not move.}
  \label{fig:algorithm-steps}
\end{figure*}

Every $B\in\mathcal B^t$ is a set of $t$ sorted breakpoints
$b_1<\cdots<b_t$, each with a corresponding function
$f_i\in\mathcal F$. Now, we can use these breakpoints to define
piecewise quadratic functions.
\begin{definition}[Storage for piecewise quadratic function with $t$ breakpoints]
  For any index $J\in\{1,\dots, t+1\}$, function $M\in\mathcal F$,
  breakpoints $B\in\mathcal B^t$, we define
$$ 
P_{J,M,B}(\mu) = M(\mu)
-\sum_{i=1}^{J-1} f_i(\mu) I[\mu < b_i]
+\sum_{i=J}^t     f_i(\mu) I[b_i < \mu],
$$
where $I$ is the indicator function (one if true and zero otherwise).
\end{definition}
This definition assumes that $M(\mu)$ is the $J$-th function piece,
and that each function $f_i$ is the difference between cost functions
on either side of breakpoint $b_i$ (see
Figure~\ref{fig:algorithm-steps}). More precisely, for all
$i\in\{1,\dots,t+1\}$, let $c_i\in\mathcal F$ be the cost function
between breakpoints $b_{i-1}$ and $b_i$ (using the convention
$b_0=-\infty$ and $b_{t+1}=\infty$). Then
$c_i(\mu)+f_i(\mu)=c_{i+1}(\mu)$ for all breakpoints
$i\in\{1,\dots,t\}$.
\begin{example}
  The positive part function is a piecewise quadratic function with
  $t=1$ breakpoint. Let $B=\{(0,\mu\mapsto\mu)\}$ be the set which
  contains one breakpoint. The cost before the breakpoint is
  $c_1(\mu)=0$ and the cost after the breakpoint is
  $c_2(\mu)=\mu$. The positive part function $(\mu)_+$ is equivalent
  to $P_{1,c_1,B}(\mu)=0+\mu I[0 < \mu]$ and
  $P_{2,c_2,B}(\mu)=\mu - \mu I[\mu < 0]$.
\end{example}
%\begin{theorem}[Dynamic programming computation of total hinge loss]
\subsection{Dynamic programming computation of total hinge loss}
The total hinge loss is a piecewise quadratic function with $t$
breakpoints $\underline C_t(\mu)=P_{J_t,M_t,B_t}(\mu)$, where
$B_t,J_t,M_t$ can be recursively computed using the following dynamic
programming update rules (this is a claim which needs proof). The
initialization is
\begin{equation}
  \label{eq:init}
  B_0=\{\},\, J_0=1,\, M_0(\mu)=0
\end{equation}
The set of breakpoints and corresponding difference functions is
$B_t\in\mathcal B^t$. Its update rule is to insert the new breakpoint
and difference function:
\begin{equation}
  \label{eq:B_t}
  B_t=B_{t-1}\cup \{(y_t-s_t\epsilon,\, s_t\ell[s_t(\mu-y_t)+\epsilon])\}
\end{equation}
The pointer before optimization is $j_t\in\{1,\dots,t+1\}$. It is
updated by adding one if the new breakpoint is before the old
pointer. The notation $b_{t,i}\in\mathbb R$ means the $i$-th
breakpoint in $B_t$.
\begin{equation}
  \label{eq:j_t}
  j_t = J_{t-1} + I[y_t-s_t\epsilon < b_{t-1,J_{t-1}}]
\end{equation}
The cost before optimization is $m_t\in\mathcal F$. It is updated by adding the
cost of the new data point, if it is positive at the pointer:
\begin{equation}
  \label{eq:m_t}
  m_t(\mu) = M_{t-1}(\mu) + 
  \ell[s_t(\mu-y_t)+\epsilon]
  I[0<s_t(b_{t,j_t}-y_t)+\epsilon]
\end{equation}
The pointer after optimization is $J_t\in\{1,\dots,t+1\}$. It is
updated by moving it to the right of the interval that contains the
new minimum (see Figure~\ref{fig:algorithm-steps}). The notation
$c_{t,i}\in\mathcal F$ means the cost function of the $i$-th interval
(between $b_{t,i-1}$ and $b_{t,i}$), which are not all stored at
once. However, we know that $c_{t,j_t}(\mu)=m_t(\mu)$ and since the
cost function is convex, we need only move the pointer to the left or
the right until we find the global minimum:
\begin{equation}
  \label{eq:J_t}
  J_t = \argmax_{i\in\{1,\dots,t+1\}}
  |(b_{t,t-1}, b_{t,i}]\cap \argmin_{\mu\in\mathbb R}\underline C_t(\mu)|
\end{equation}
The cost after optimization is $M_t\in\mathcal F$, and it is updated every time
the pointer is moved, by adding or subtracting an $f_{t,i}$ function
(the function stored in $B_t$ at $b_{t,i}$).
\begin{equation}
  \label{eq:M_t}
  M_t(\mu)=m_t(\mu) +
  \begin{cases}
    0 & \text{ if } j_t = J_t\\
    \sum_{i=j_t}^{J_t-1} f_{t,i}(\mu) & \text{ if } j_t < J_t\\
    -\sum_{i=J_t}^{j_t-1} f_{t,i}(\mu) & \text{ if } J_t < j_t
  \end{cases}
\end{equation}

\subsection{Proof of optimality}

We first would like to show that the cost we want to minimize
$\underline C_t(\mu)$ at time $t$ can be written in terms of the cost
functions $c_{t,i}(\mu)$ on each interval $i\in\{1,\dots, t+1\}$
between breakpoints.

\begin{definition}
  We define the cost at time $t$ on interval $i$ as
  \label{def:cti}
  \begin{equation*}
    c_{t,i}(\mu)=\sum_{j=1}^t
    \ell[s_j(\mu-y_j)-\epsilon]
    I[(s_j=-1\cap b_{t,i-1}<y_j+\epsilon)\cup(s_j=1\cap y_j-\epsilon<b_{t,i})]
  \end{equation*}
\end{definition}

\begin{lemma}
  \label{lemma:cost-intervals}
  The cost up to data point $t$ can be written in terms of the cost on each interval:
$$
\underline C_t(\mu) = \sum_{i=1}^{t+1}
c_{t,i}(\mu) I[b_{t,i-1}\leq \mu \leq b_{t,i}].
$$
\end{lemma}

\begin{proof}
  First assume that at time $t$ the new breakpoint
  $b_{t,j}=y_t-s_t\epsilon$ is $j$-th in the sequence
  $b_{t,1}<\cdots<b_{t,j}<\cdots<b_{t,t}$. Based on update rule
  (\ref{eq:B_t}), we can then write the breakpoints at the previous time
  $t-1$ as
  \begin{equation}
    b_{t-1,i} =
  \begin{cases}
    b_{t,i} & \text{ for } i\in\{1,\dots,j-1\} \\
    b_{t,i+1}&\text{ for } i\in\{j,\dots,t\}.
  \end{cases}
  \end{equation}
  And we can write the breakpoints at time $t$ in terms of the previous breakpoints:
  \begin{equation}
      b_{t,i} =
  \begin{cases}
    b_{t-1,i}&\text{ for }i\in\{1,\dots,j-1\} \\
    y_t-s_t\epsilon&\text{ for }i=j\\
    b_{t-1,i-1}&\text{ for }i\in\{j+1,\dots,t+1\}
  \end{cases}
  \label{eq:recursive-b}
  \end{equation}
  We proceed by induction. First we prove that the equality holds for
  $t=1$. There are two cases. For a lower limit $s_i=-1$, we have
  \begin{eqnarray}
\underline C_1(\mu)&=&
\begin{cases}
  c_{1,1}(\mu)=\ell[s_1(\mu-y_1)+\epsilon] &\text{ if }-(\mu-y_1)+\epsilon>0\Rightarrow\mu < b_{1,1}=y_1+\epsilon\\
  c_{1,2}(\mu)=0 & \text{ otherwise}
\end{cases}
  \end{eqnarray}
For an upper limit $s_i=1$, we have
  \begin{eqnarray}
\underline C_1(\mu)&=&
\begin{cases}
  c_{1,2}(\mu)=\ell[s_1(\mu-y_1)+\epsilon] &\text{ if }(\mu-y_1)+\epsilon>0\Rightarrow\mu > b_{1,1}=y_1-\epsilon\\
  c_{1,1}(\mu)=0 & \text{ otherwise}
\end{cases}
  \end{eqnarray}
  For both cases it is clear that Lemma~\ref{lemma:cost-intervals}
  is true for $t=1$. Now we assume that it is true for $t-1$, and
  prove that it is true for $t$. First assume that $s_t=-1$, which
  implies from Definition~\ref{def:cti} that 
  \begin{equation}
    c_{t-1,i}(\mu)=
    \begin{cases}
      c_{t,i}(\mu)-\ell[s_t(\mu-y_t)+\epsilon] & \text{ for } i\in\{1,\dots, j-1\}\\
      c_{t,j+1}(\mu)=c_{t,j}(\mu)-\ell[s_t(\mu-y_t)+\epsilon] & \text{ for } i=j\\
      c_{t,i+1}(\mu) & \text{ for } i\in\{j+1, \dots, t\}.
    \end{cases}
  \label{eq:recursive-c}
  \end{equation}
and
\begin{equation}
  c_{t,i}(\mu)=
  \begin{cases}
    c_{t-1,i}(\mu)+\ell[s_t(\mu-y_t)+\epsilon] & \text{ for }i\in\{1,\dots,j\}\\
    c_{t-1,i-1}(\mu) & \text{ for }i\in\{j+1,\dots,t+1\}.
  \end{cases}
\end{equation}
We then start from the induction hypothesis at $t-1$, and derive an expression in terms of $t$:
\begin{eqnarray}
  \underline C_{t-1}(\mu) 
  &=& \sum_{i=1}^{t} c_{t-1,i}(\mu)I[b_{t-1,i-1}\leq\mu\leq b_{t-1,i}]\\
  &=& \sum_{i=1}^{j-1} c_{t-1,i}(\mu)I[b_{t-1,i-1}\leq \mu\leq b_{t-1,i}]+\nonumber\\
      &&c_{t-1,j}(\mu)I[b_{t-1,j-1}\leq\mu\leq b_{t-1,j}] +\nonumber\\
      &&\sum_{i=j+1}^{t} c_{t-1,i}(\mu)I[b_{t-1,i-1}\leq \mu\leq b_{t-1,i}]\label{eq:separate}\\
  &=& \sum_{i=1}^{j-1} \big(c_{t,i}(\mu)-\ell[s_t(\mu-y_t)+\epsilon]\big)I[b_{t,i-1}\leq\mu\leq b_{t,i}]+\nonumber\\
      &&\big(c_{t,j}(\mu)-\ell[s_t(\mu-y_t)+\epsilon]\big)I[b_{t,j-1}\leq\mu\leq b_{t,j}] +\nonumber\\
      &&c_{t,j+1}(\mu)I[b_{t,j}\leq\mu\leq b_{t,j+1}] +\nonumber\\
      &&\sum_{i=j+1}^{t} c_{t,i+1}(\mu)I[b_{t,i}\leq\mu\leq b_{t,i+1}]\label{eq:recursive-c-b}\\
  &=& \sum_{i=1}^{j} \big(c_{t,i}(\mu)-\ell[s_t(\mu-y_t)+\epsilon]\big)I[b_{t,i-1}\leq\mu\leq b_{t,i}]+\nonumber\\
      &&\sum_{i=j}^{t} c_{t,i+1}(\mu)I[b_{t,i}\leq \mu\leq b_{t,i+1}]
\end{eqnarray}
Equation (\ref{eq:separate}) results from taking the $j$ term out of
the sum, and (\ref{eq:recursive-c-b}) results from applying recursive
formulas for $c_{t,i}$ (\ref{eq:recursive-c}) and $b_{t,i}$
(\ref{eq:recursive-b}). We then can prove the equality for $t$:
\begin{eqnarray}
\underline C_t(\mu)
&=& \sum_{i=1}^t \phi[s_i(\mu-y_i)+\epsilon]\\
&=& \phi[s_t(\mu-y_t)+\epsilon] + \underline C_{t-1}(\mu)\\
&=& \ell[s_t(\mu-y_t)+\epsilon] I[\mu<y_t+\epsilon]+\underline C_{t-1}(\mu)\\
&=& \ell[s_t(\mu-y_t)+\epsilon] I[\mu<b_{t,j}]+\nonumber\\
 &&\sum_{i=1}^{j} \big(c_{t,i}(\mu)-\ell[s_t(\mu-y_t)+\epsilon]\big)I[b_{t,i-1}\leq\mu\leq b_{t,i}]+\nonumber\\
      &&\sum_{i=j}^{t} c_{t,i+1}(\mu)I[b_{t,i}\leq \mu\leq b_{t,i+1}]\\
 &=&\sum_{i=1}^{j} c_{t,i}(\mu)I[b_{t,i-1}\leq\mu\leq b_{t,i}]+\nonumber\\
      &&\sum_{i=j}^{t} c_{t,i+1}(\mu)I[b_{t,i}\leq \mu\leq b_{t,i+1}]\\
 &=&\sum_{i=1}^{t+1} c_{t,i}(\mu)I[b_{t,i-1}\leq\mu\leq b_{t,i}].
\end{eqnarray}
This concludes the proof of Lemma~\ref{lemma:cost-intervals} for the
case of $s_t=-1$, and the case for $s_t=1$ is analogous.
\end{proof}

\begin{theorem}
  The optimization problem can be solved using
  $\min_\mu\underline
  C_t(\mu)=\min_{\mu\in(b_{t,J_t-1},b_{t,J_t}]}M_t(\mu)$.
\end{theorem}

\begin{proof}
  TODO
\end{proof}

\subsection{Implementation details and complexity analysis}

We propose to store each quadratic function $f\in\mathcal F$,
$f(\mu)=a\mu^2 + b\mu + c$ in terms of its three coefficients
$a,b,c\in\mathbb R$. Update rule (\ref{eq:m_t}) can thus be
implemented in constant $O(1)$ time, by simply adding the coefficients of
$M_{t-1}(\mu)$ and $\ell[s_t(\mu-y_t)+\epsilon]$
(line~\ref{line:add-coefs} of Algorithm~\ref{algo:dp}). Note that
$B[J].\text{breakpoint}$ in Algorithm~\ref{algo:dp} is the 
breakpoint at the pointer ($b_{t,j_t}$ in equation~\ref{eq:m_t}).

We propose to store the set of breakpoints $B_t$ using the map
container of the C++ Standard Template Library ($B$ in
Algorithm~\ref{algo:dp}). It guarantees that the insert breakpoint
operation of update rule (\ref{eq:B_t}) takes $O(\log t)$ time
(line~\ref{line:insert}). The pointers $j_t$ and $J_t$ can be implemented
using a map::iterator ($J$ in Algorithm~\ref{algo:dp}). So update rule
(\ref{eq:j_t}) happens automatically when the new breakpoint is
inserted -- the variable $J$ is $J_{t-1}$ before the insert, and it
becomes $j_t$ after the insert.

Update rules for $J_t$ (\ref{eq:J_t}) and $M_t$ (\ref{eq:M_t}) are
implemented in the while loop on
lines~\ref{line:while}--\ref{line:else} of
Algorithm~\ref{algo:dp}. Each iteration of the while loop is a
constant $O(1)$ time operation. The MinInInterval sub-routine exploits
the convexity of the cost function, and returns TRUE if the minimum of
$M_t(\mu)$ occurs between $b_{t,J-1}=B[J-1].\text{breakpoint}$ and
$b_{t,J}=B[J].\text{breakpoint}$. This can be implemented by testing
the derivative of $M$ at the limits, by using the fact that
$M_t(\mu)=c_{t,J}(\mu)$ for the current pointer $J$. If the slope is
positive on the left $0<c_{t,J}'(b_{t,J-1})$, then $M_t$ is increasing
in the interval, and the pointer should be moved left
(line~\ref{line:increasing} of Algorithm~\ref{algo:dp}). If the slope
is negative on the right $c_{t,J}'(b_{t,J})<0$, then $M_t$ is
decreasing in the interval, and the pointer should be moved left
(line~\ref{line:else} of Algorithm~\ref{algo:dp}).

Once the pointer has been moved to the interval that contains the
minimum, the Minimize sub-routine returns an optimal prediction
$\mu_t^*$ and cost $C_t^*$ (line~\ref{line:Minimize}) in constant
$O(1)$ time.

Overall, the complexity of Algorithm~\ref{algo:dp} is $O(n)$ space and
$O(n (m + \log n))$ time, where $m$ is the average number of pointer moves
(the while loop, line~\ref{line:while}). TODO: explore worst-case
complexity with pathological data sets -- can we get a data set which
has $O(t)$ pointer moves, resulting in $O(n^2)$ overall time
complexity? For the linear hinge loss $m=O(1)$, TODO proof. For the
squared hinge loss we show empirically that the number of moves is
constant. The overall empirical time complexity is thus $O(n\log n)$.

\begin{algorithm}[H]
\begin{algorithmic}[1]
\STATE Input: 
limits $\mathbf y\in\mathbb R^n$, 
signs $s\in\{-1,1\}$, 
margin $\epsilon\in\mathbb R$.
\STATE Initialize: 
$B\gets\text{map}\{\},\ J\gets B.\text{end}(),\ M\gets\text{Coefs(0)}$
\STATE for data points $t$ from 1 to $n$:
\begin{ALC@g}
  \STATE $f\gets\text{Coefs}[s_t\ell(s_t(\mu-y_t)+\epsilon)]$
  \STATE $b\gets y_t - s_t\epsilon$
  \STATE $B.\text{insert}(b, f)$ \label{line:insert}
  \STATE if $0 < s_t(B[J].\text{breakpoint}-y_t)+\epsilon$:
  \begin{ALC@g}
    \STATE $M$ += $\text{Coefs}[\ell(s_t(\mu-y_t)+\epsilon)]$
    \label{line:add-coefs}
  \end{ALC@g}
  \STATE while $!\text{MinInInterval}(M,B,J)$:
  \label{line:while}
  \begin{ALC@g}
    \STATE if $\text{Increasing}(M)$: $J--$; $M$ $-$= $B[J].\text{function}$
    \label{line:increasing}
    \STATE else: $M$ += $B[J].\text{function}$; $J++$
    \label{line:else}
  \end{ALC@g}
  \STATE $\mu_t^*, C_t^*\gets\text{Minimize}(M,B,J)$
  \label{line:Minimize}
\end{ALC@g}
\STATE Output: $\mu^*\in\mathbb R^n, C^*\in\mathbb R^n$
\caption{\label{algo:dp}Dynamic programming algorithm for
  computing minimum total hinge loss.}
\end{algorithmic}
\end{algorithm}

\bibliographystyle{natbib}
\bibliography{refs}

\end{document}

